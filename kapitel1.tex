\chapter{Grammatiken}
\pstree[levelsep=8ex]{\Tcircle{R} }{
\Tcircle{D1}
\pstree{ \Tcircle{D2} }{
\Tcircle{D2.1}
\Tcircle{D2.2}
}
\Tcircle{D3}
\Tcircle{D4}
}

\section{Symbole}
Symbole sind alle verwendbaren Zeichen eines Alphabets.

\section{Alphabet}
Ein Alphabet, abgek�rzt $\Sigma$, ist eine endliche Menge von einzelnen Symbolen. Ein Alphabet wird als endliche Menge von Symbolen definiert.

$\Sigma = \{\smiley, a, Anton\}$

Neben einzelnen Buchstaben k�nnen auch W�rter im Sinne der deutschen Sprache verwendet werden. Es k�nnen aber auch andere Symbole (\smiley) verwendet werden. Bei Verwendung von W�rtern als Symbole eines Alphabets ist aber zu beachten, dass bei der sp�teren Verwendung die einzelnen Buchstaben oder Teile des Wortes nicht als Symbole genommen werden d�rfen, sondern das gesamte Wort als ein Symbol bezeichnet wird. Im hier verwendeten Fall darf also das gro�e $A$ nicht als Symbol verwendet werden darf. ebenso d�rfen auch keine Wortteile wie $ton$ nicht verwendet werden. Ein zul�ssiges Symbol ist allerdings das gesamte Wort $Anton$.

\section{W�rter}
W�rter werden aus Kombination von einzelnen Symbolen gebildet. Die bildbaren W�rter werden in einer Menge $\Sigma*$ gesammelt. Wird die Menge aller W�rter gebildet muss jetzt explizit das leere Wort $\epsilon$ angegeben werden. Auf Grund der in der Grammatik verwendeten Regeln wird die Menge der m�glichen W�rter eingeschr�nkt. In unserem Beispiel kann die Menge aller W�rter so aussehen:

$\Sigma* = \{\epsilon, \smiley, a, Anton, \smiley a, a Anton, \smiley Anton, a \smiley, Anton a, Anton \smiley, \smiley \smiley, ...\}$

Die Menge wird gebildet, in dem erst das leere Wort, dann alle einstelligen W�rter, alle zweistelligen W�rter, alle dreistelligen W�rter usw. in die Menge integriert werden. Die Menge der W�rter ist unendlich, da die L�nge der W�rter nicht bestimmt ist.

Die L�nge eines Wortes $|w|$ bezeichnet die Anzahl der f�r das Wort verwendeten Symbole.

\begin{figure}[h]
	$|\epsilon| = 0$
	
	$|a| = 1$
	
	$|Anton| = 1$
	
	$|aa| = 2$
	\caption{Beispiele f�r Wortl�ngen}
\end{figure}

\section{Grammatiken}
Eine Grammatik $G$ wird definiert durch:
\begin{itemize}
	\item eine Menge von Variablen $V=\{S,U\}$, die bei der Ableitung verschwinden,
	\item eine Menge von Terminalsymbolen $\Sigma$, gelichbedeutend mit dem definierten Alphabet
	\item ein Startsymbol $S \in V$
	\item und eine Menge von Regeln $P$
\end{itemize}

Formal wird eine Grammatik also beschrieben als:

$G = (V, \Sigma, S, P)$

\subsection{Regeln}
Eine Regel besteht aus einer linken und einer rechten Regelseite. Bei der Ableitung von Regeln wird der Ausdruck auf der linken Seite durch den Ausdruck auf der rechten Seite ersetzt.

Bei der Regel $S \rightarrow Anton$ wird die Variable $S$ durch das Symbol $Anton$ ersetzt. 

\subsection{Ableitung}
Bei der Ableitung von Regeln beginnt man beim Startsymbol und ersetzt das Startsymbol durch den entsprechenden Ausdruck einer Regel, die auf das Startsymbol passt. Zwischen zwei Ableitungsschritten kommt dann ein Doppelpfeil ($\Rightarrow$).

\begin{figure}[h]
$S \rightarrow Anton$ \textbf{(Regel)}

$S \Rightarrow Anton$ \textbf{(Ableitung)}
\caption{Einfaches Beispiel f�r eine Ableitung}
\end{figure}

\subsection{Regelalternativen}
Regeln k�nnen auch Alternativen enthalten, um mehr M�glichkeiten zu bieten W�rter zu erzeugen.
Alternativen werden auf der rechten Regelseite durch ein Pipe-Symbol ($|$) eingeleitet.
Eine Regel mit Alternativen k�nnte folgenderma�en aussehen:

$S \rightarrow Anton | Anton \smiley | Anton a$

Dadurch w�ren die W�rter $Anton$, $Anton \smiley$ und $Anton a$ f�r die Variable $S$ m�glich.